\chapter{Conclusion}
In this final chapter a conclusion on how well the final product works and which of the requirements.
The user will have to use commands to navigate through inputs given to the program. The program then reads the input and uses the input for its purpose and decides what actions should be taken via the inputs. The program will write text to the user that explains what the user needs to know with the different steps through the program.

Then there is the problem that the program is a console application and not a windows application. Therefore one of the main problems is to make the program as user friendly as possible. This problem has been managed at a reasonable point, with text that helps and guides the user and if the user is failing in the process, the program can handle it in most cases.

The functions that are used in the program in order to handle the jpg photo come from a library named "libJPEG". This library helps the program handle images and edit them to the right size in terms of height and width or passport photos. Then the program asks a series of questions to check that the other rules are covered. To check if the photo is of high enough quality there has been made a function that can check the quality of the photos.

The program can open an image, and are able to draw a grid to help the user with the input coordinates. The program can then crop the image with the input from the user and show the crop image, so the user can see if the image is cropped right.
The program then checks the DPI and brightness of the image and will inform the user if any of these two the do not meet the restrictions. The user is then to answer a list of questions regarding the photo to ensure that all rules for passport photo are followed.
After the questions the program saves the edited image which then should be possible to use as a passport photo. The program informs the user what to do all the way through the process and guides the user through the rules.

So the program covers the core requirements that have been developed for the product and thereby the program should help solve the problem set in the thesis statement. 
It can be conclude that the program gives a product that should be more reliable as passport photo rather than just a photo taken by the user themselves.
Maybe there are small problems with the interface but that does not play any role regarding the product that the program gives the user.
Therefore it is a good program that helps the user with the process but it can be improved by using another language that makes it easier to make use of an interface.