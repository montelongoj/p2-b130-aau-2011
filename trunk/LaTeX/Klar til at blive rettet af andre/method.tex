\section{Method}
This project structure will be based on Aalborg PBL (problem based learning). The Aalborg PBL is a method whereby the learning process lies in the work with problem and try to develop a solution for the given problem.
The Aalborg PBL method also trains the students ability to work together in a project group and give them tools to handle the processes that goes with working in a group.

The first stage of the project is the problem analysis, which purpose is to find and document a that there are a problem to begin with. From the problem analysis a thesis statement are formed and are used to produce a list of product requirements.
The requirements are then use to design and develop a product that should solve the problem stated in the thesis statement.
The program are then tested on the targeted group that was found affected by the problem. The testing leads to a conclusion of the project. This is main course of the project work when using the Aalborg PBL.
This project form is used because it first finds and document a problem and then through the work with the problem gives an estimated solution to the problem.\\

To document the problem a lot of information is needed. The needed information is found through different sources as books, article, websites and etc. When using information found through the internet or other sources it is important to evaluate the used sources.
This is done to filter out none reliable sources and thereby achieve a better and more trustworthy project.
This process of evaluation is also known as source criticism and are general used when using others materials as documentation.

Therefore it is also a relevant method to use when using sources in the project work.\\
