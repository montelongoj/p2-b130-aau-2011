\chapter{Retrospective}
In this chapter we will discuss, put this project into perspective, and the work regarding it.
The working time of the project was from 9:10 to 16:15 every day given that no lecture was scheduled that day. By doing this there was a high productivity in the group, and there was still was time for fun. To improve the productivity, the option to work at home from 14:00 was removed when we went behind schedule.
We did this because working in the group room resulted in a higher productivity. One of the good features for a group is that we are social and friendly with each other because it enhances and improves the group spirit. This can be useful for work in the future that you can work with other people and be part of a group.

We had a meeting with the supervisor nearly every week to make sure that we were on the right track with the project all the time. There was a single time when the supervisor meeting was pointless because the week before there had only been work on the program. The overview of our project was easy to keep since the structure of the report was well organized. The schedule was well made and that made it easy to keep track of the progress of the project. In this project there were a few roles in the group. The roles were those we thought were necessary to ensure optimal group work. This time there was only made a summary at the meetings with the supervisor and not for the morning meetings since it was not used that much as in the previous project. 

Compared to our P1 project, we had to work somewhat harder with the result being a better report and a good product. Our working progress has been nearly the same, but since we got behind the schedule for this project we had to work a little harder. The method for project work makes the process easier to comprehend and thus making it good for future projects.