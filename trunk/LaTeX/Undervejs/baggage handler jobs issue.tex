\subsection{Baggage handler jobs issue}
The following part is based on a report with the name "Musculoskeletal ill-health risks for airport baggage handlers". It is made for "East Midlands Airport" that is prepared by "Health and Safety Laboratory" and is made in 2009.


The current luggage handling methods that is in use shows some manual handling, which have a high risk of injury. These methods include "vertical height and distance of the lift, the hand reach from the lower back, trunk twisting and sideways bending, postural constraints and the grip on the load. The potentially high load weights combined with the frequency of handling also contribute to the risk of injury" \citep{}.


To have a better look at what different methods there are in use. There will be a short overview of some of them here:


1. Extending Belt Loader (EBL) type technology \newline
This method helps the worker by reducing the risk of postural improvement. The EBL has the function to adjust the height and extend the belt. But if the function is not there it has the same risk as a normal belt loader. Still there is a major risk reduction for the internal worker because of the automation of the transfer of the bags down the ship hold, thus reduce the lifting time and posture when lifting.


2. Beltloader \newline
then using the Beltloader the vertical lift region is generally above knee and below shoulder level. If set op in a bad way there may be an higher possibility for twisting and/or sideways bending with the baggage. But one need to move the baggage manually through the plans hold.


%3. Mallaghan LBT90 \newline


3. Direct to hold from ground level \newline
The distance baggage needed to be lifted op can depend on the distance from the cart and the cargo-hold door on the plane. And there is a high chance of risk when reaching to the opposite side of the cart.

4. Direct to hold using a flatbed truck \newline
the hight that the baggage handler need to lift are between around the floor and the shoulders. This depend on the headroom. There might be a higher chance that the worker that take the bag into the planes cargo-hold will have a restrict posture. There is also a risk of falling from a low hight because they work on a open platform that can be slippery. Besides that the worker that is inside the plane need to move the bags down the hold. therefore this is seen as the one that have the most risks.



2.4 -  2.6















baggage handlers method they are working have a numbers of manual handling that are a high risk of injury. Those injury (can) come from "the vertical height and distance of the lift, the horizontal distance of the hands from the lower back, trunk twisting and sideways bending, postural constraints and the grip on the load. The potentially high load weights combined with the 
frequency of handling also contribute to the risk of injury."
