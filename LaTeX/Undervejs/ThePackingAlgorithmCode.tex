\section{The Development of The Packing Algorithm}
\label{sec:devalgorithm}
The algorithm was developed from the theory described in section \ref{sec:algorithm}. The theory has been extended from the experiences when packing the program. These extensions have made the packing more effective, which result in that more items can be packed. 
To understand the different methods, the algorithm as described in section \ref{sec:algorithm} must be extended a bit. The extension will be describes in this section, with a roundup at the end. To describe the code, this section will be divided in different subsections to understand the different parts of the code.

\subsection{Classes}
To handle the information and functions needed to handle suitcases and items to pack, the classes luggage and luggage\_item has been made. They do both inherit from the class Cube\_Shape, with the variables width, depth, height and name. The Cube\_Shape class provides the information of a cube, which both suitcases and items can be seen as. It is the dimensions and a name. 
\subsubsection{The class "luggage"}
The class diagram of the luggage class can be seen on Figure \ref{fig:lugclass}. There are a lot of private fields (starting with "\_") which can be accessed through properties. It contains all the values needed: the maximum weight, the weight of the suitcase etc. It contains also a set of methods which is used through the packing process. The properties which should only be accessed in the methods in the class has a private setter, so functions outside the class only can get the value, and not change the properties by a mistake. An example of this is the property "weight", which indicates the total weight of the items in the suitcase and the suitcase' weight. This value should only be set by the class it self in the methods placing an item.

\figur{1}{LuggageClass.jpg}{The class diagram of the luggage class}{fig:lugclass}

As seen on Figure \ref{fig:lugclass} there are a set of methods in the class "luggage". These methods provide the necessary functions to handle the suitcase while packing.

\subsubsection{The class "luggage\_item"}
The class luggage\_item is for the items to pack in the luggage. The class diagram can be seen on Figure \ref{fig:lugitemclass}. As in the luggage class, this class has some variables, which can only be accessed through properties. Those properties, which should only be set inside the class have a private setter. There are also a private method "Rotate\_Lug\_Item", which should only be called inside the class.
\figur{1}{LuggageItemClass.jpg}{The class diagram of the luggage\_item class}{fig:lugitemclass}'

\subsection{Description of the code}
The algorithm to pack has been develloped from the flowchart on Figure \ref{fig:flowalgo}. The following will describe the code and the extensions of the flowchart.
\subsubsection{Sorting}
As seen on the figure, the first thing which is done is to sort the suitcases and the items. Before sorting the suitcases and luggages, it resets the values changes when packing. This is to ensure the algorithm has the right values when running it twice. The reset methods for the items and the suitcases are in their respective classes. 