\section{The Development of The Packing Algorithm}
\label{sec:devalgorithm}
The algorithm was developed from the theory described in section \ref{sec:algorithm}. The theory has been extended from the experiences when packing the program. These extensions have made the packing more effective, which result in that more items can be packed. To describe the code, this section will be divided in different subsections to understand the different parts of the code.
\subsection{Classes}
To handle the information and functions needed to handle suitcases and items to pack, the classes luggage and luggage\_item has been made. They do both inherit from the class Cube\_Shape, with the variables width, depth, height and name. The Cube\_Shape class provides the information of a cube, which both suitcases and items can be seen as. It is the dimensions and a name. 
\subsubsection{The class "luggage"}
The class diagram of the luggage class can be seen on figure \ref{fig:lugclass}. There are a lot of private fields (starting with "\_") which can be accessed through properties. It contains all the values needed: the maximum weight, the weight of the suitcase etc. It contains also a set of methods which is used through the packing process.

\figur{1}{LuggageClass.jpg}{The class diagram of the luggage class}{fig:lugclass}


\subsubsection{The class "luggage_item"}